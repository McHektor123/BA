

\chapter{Grundlagen zu Neo4j} % Main chapter title

\label{Kaptiel 2} % Change X to a consecutive number; for referencing this chapter elsewhere, use \ref{ChapterX}

%----------------------------------------------------------------------------------------
%	SECTION 1
%----------------------------------------------------------------------------------------
\section{Graph als Datenstruktur}
Ein Graph ist eine abstrakte Datenstruktur, welche aus der Mathematik stammt \parencite{vicknair2010comparison}. Graphen werden als ein geordnetes Triple (V(G), E(G), $\psi_G$) aufgefasst, für GBDs sind V(G) und E(G) endliche Mengen. V(G) ist eine nicht leere Menge von Knoten, auch Punkte genannt. E(G) ist eine Menge von Kanten. Die Funktion $\psi$ weist jeder Kante ein Tupel aus Knoten zu, so stellt $\psi_G (e) = (v_1 v_2)$ eine Verbindung der Knoten $v_1$ und $v_2$ durch die Kante $e$ dar. Wenn $\psi$ ein geordnetes Tupel von Knoten verwendet und die Reihenfolge somit relevant ist, besitzt die Kante  eine  Richtung und der Graph wird als gerichtet bezeichnet, bei einem ungeordneten Paar wird von einem ungerichteten Graph gesprochen. Wenn die Kanten Gewichte oder Kosten zur Traversierung besitzen, wird der Graph als gewichtet bezeichnet und ohne Gewichte als ungewichtet \parencite{bondy1976graph}.

\section{Temporale Datenbanken}
Eine temporale Datenbank besitzt die Fähigkeit, alle Daten in Abhängigkeit von einer zeitlichen Einheit zu speichern \parencite{campos2016towards}. Jede Relation und Entität kann ein Datum besitzen, zu welchem diese gültig ist. Zu jedem Zeitpunkt, der nicht diesem Datum entspricht, ist es nicht garantiert, dass der betrachtete Eintrag in der Datenbank gültig ist. Beispiel 1: Wenn eine Person zu einem genannten Zeitpunkt bei einer Firma angestellt ist, kann zu einem Zeitpunkt, der nach dem genannten Zeitpunkt folgt, keine garantierte Aussage darüber getroffen werden, ob die Person noch bei der Firma angestellt ist. \newline
Die Objekte und Relationen können Attribute von einem zeitlichen Datentypen besitzen, welches eine bestimmte Eigenschaft beschreibt und nichts über die Gültigkeit aussagt \parencite{khurana2012introduction}.  
%-----------------------------------
%	SUBSECTION 1
%-----------------------------------
\section{Überblick zu Neo4j}
Neo4j ist eine in Java implementierte temporale Graphdatenbank \parencite{vukotic2015neo4j}. Als grundlegende Datenstruktur wird ein gerichteter und gewichteter Graph verwendet. Knoten stellen die Entitäten dar und  Kanten stellen die Relationen zwischen den Entitäten dar.  Attribute werden als zusätzliche Informationen in den Knoten oder Katen gespeichert, wie $Name$ bei einem Knoten vom Typ $Person$. Knoten und Kanten können mit Bezeichnern versehen werden, um so leichter in Anfragen  verwendet werden zu können. Die zur Verfügung stehenden Operationen von Neo4j sind entweder durch die jeweilige Programmiersprache oder durch die  Anfragesprache Cypher definiert, wobei Cypher eine standardisierten Syntax mit mehreren vordefinierten Funktionen besitzt. Es wird das Einbetten weiterer Bibliotheken unterstützt, welche  zusätzliche Funktionen zur Verfügung stellen. Durch diese Funktionen ist es unter anderem möglich Daten aus verschiedenen Formaten wie JSON, CSV oder XML in die Datenbank zu laden oder Daten aus einer anderen Web-API (Application programming interface) zu nutzen. Neo4j lässt sich im eingebetteten Modus oder im  Server-Modus nutzen. Der eingebettete Modus dient der direkten  Nutzung durch die Java Core API von Neo4j. Der Server-Modus ermöglicht die parallele Nutzung auf mehreren Systemen. 

\section{Neo4j als Datenbankmanagementsystem}
Ein Datenbankmanagementsystem (DBMS) ist für die Verarbeitung von Anfragen verantwortlich und kann in folgende Teilsysteme unterteilt werden\parencite{angles2012comparison}:
\begin{enumerate}
	\item Schnittstellen für den Nutzer
	\item Eine Anfragesprache
	\item Ein Anfrage-Optimierer
	\item Eine Speicherverwaltung
	\item Eine Transaktionseinheit
	\item Eine Database Engine
	\item Operationsmöglichkeiten zum Wiederherstellen von Daten
\end{enumerate}
Die Teilsysteme werden durch die in der Abbildung \ref{fig:Architecure} dargestellten Architektur realisiert.
\begin{enumerate}
	\item Durch die Schnittstellen ist es dem Nutzer möglich mit den System zu agieren und Daten zu manipulieren. Als Schnittstellen für den Nutzer stehen die Webanwendung Neo4j Browser und Desktopanwendung Neo4j Desktop zur Verfügung. Zusätzlich besteht die Möglichkeit, eine von den Neo4j-Treibern unterstützte Programmiersprache  oder  Java zu verwenden und direkt die Neo4j API zu nutzen. 
	\item
	Die Anfragesprache  beschreibt die Sprache in welcher der Nutzer seine Anfragen an das System stellt. Für Neo4j Browser und Neo4j Desktop ist Cypher die Anfragesprache.
	\item Der Optimierer beschleunigt durch Ausführung von mehreren Teilschritten das Bearbeiten von Anfragen. Als Optimierer für Anfragen in Cypher wird der Cypher Query Optimizer verwendet\parencite{Optimizer}. 
	\item Die Speicherverwaltung dient dem Ablegen der Daten im physischen Speicher. In Neoj werden die Record Files hierfür verwendet.
	\item Die Transaktionseinheit stellt sicher, dass keine Fehler während dem Ausführen von Transaktionen geschehen. 
	\item Die Database Engine bildet das Gesamtsystem der einzelnen Komponenten.\item Die Operationsmöglichkeiten zum Wiederherstellen von Daten werden mit dem Transaktions-Log realisiert.
\end{enumerate}
Optional kann zur  Performanzsteigerung ein Cache verwaltet werden. Dieser Cache hält einen Teil der Datenbank in dem Hauptspeicher und ermöglicht einen schneller Zugriff auf diese Teile der Daten. Die vorgestellten Gegenstände werden in den folgenden Abschnitten erläutert. In der Neo4j Enterprise Version ist es möglich, das DBMS auf mehrere Systeme in einem Netzwerk mittels hoher Verfügbarkeits Funktion zu verteilen \parencite{vukotic2015neo4j}.

\begin{figure}[H]
	\centering
	\includegraphics [width=12cm, height=8cm]{Figures/new_architecture}
	\caption[Architekur von Neo4j]{ aus https://dzone.com/articles/graph-databases-for-beginners-native-vs-non-native(26.08.19)}
	\label{fig:Architecure}
\end{figure}

\subsection{Cypher und APIs für Neo4j}
Im Folgenden werden die Möglichkeiten zum Programmieren mit Neo4j beschrieben. Da es über die Neo4j Treiber möglich ist, Neo4j in mehreren Programmiersprachen zu nutzen und die Anzahl der unterstützten Sprachen stetig steigt, werden im folgenden nur die Kernfunktionen in Java beschrieben und verwendet. 
\subsubsection{Cypher}
Im Gegensatz zu den relationalen Datenbanken gibt es bei Graphdatenbanken keine standardisierte Anfragesprache, welche in den meisten Graphdatenbanken Verwendung findet \parencite{han2011survey}. In Neo4j besteht seit dem Jahr 2000 die Möglichkeit, die deklarative Anfragesprache Cypher zu verwendenden  \parencite{francis2018cypher}. Cypher wird von Neo4j Inc. entwickelt und wurde ursprünglich ausschließlich für die Neo4j Datenbank verwendet. Für die APIs von Neo4j sind Kenntnisse in der Programmiersprache Java bzw. einer durch die Neo4j-Treiber unterstützten Programmiersprachen notwendig. Cypher bildet eine Möglichkeit ohne diese Kenntnisse die  Datenbank anzusteuern \parencite{vukotic2015neo4j}. In Cypher wird ein Muster durch den Nutzer angegeben und alle Objekte, die dieses Muster erfüllen, werden zurückgegeben. Die wichtigsten  Prädikate sind: \newline
\textbf{Create}: Erzeugt ein Objekt in der Datenbank. 
\begin{Verbatim}[frame=single]
CREATE (p:Person)
\end{Verbatim}
Es ist möglich, mehrere Knoten mit den dazugehörigen Relationen zu erzeugen. Indizes für die Objekte oder Attribute von Objekten können ebenfalls über Create erzeugt werden.\newline
\textbf{Delete}: Wie in SQL  wird ein  oder mehrere Objekte aus der Datenbank entfernt.
\begin{Verbatim}[frame=single]
DELETE (p:Person{name: 'Peter'})  
\end{Verbatim}
\textbf{Where}: Wie in SQL werden Objekte anhand von Attributen gefiltert. \newline
\textbf{Match}: Spezifiziert das Muster in dem Neo4j sucht.
\begin{Verbatim}[frame=single]
MATCH (p1:Person)-[:Friends*2]->(p2:Person) 
WHERE p1.name= ‘Peter’ 
RETURN p2.name
\end{Verbatim}
Hier werden alle Personen-Knoten durchsucht, die mit einer Friends-Relation mit dem Personen-Knoten von Peter verbunden sind. p1 stellt einen Bezeichner für den Knoten vom Typ Person dar, [:Friends] ist eine Relation vom Typ Friends und durch “->” wird angegeben, dass es sich um eine ausgehende Kante vom Knoten p1 handelt. Mit der Schreibweise [:Friends*2] wird ausgedrückt, dass es sich um die Tiefe 2 handelt, das heißt es werden Freunde von Freunden gesucht. \newline
\textbf{Return}: Es wird angegeben, welche Objekte bzw. welche Attribute der Objekte, die das Muster erfüllen zurückgegeben werden sollen.\newline
\textbf{With}: Dadurch lassen sich in einer Anfrage Objekte manipulieren bevor sie zu einer weiteren Anfrage gegeben werden. 
\begin{Verbatim}[frame=single]
MATCH (p:Person{name: ‘Peter’})  
WITH COUNT(p) as count  
RETURN count
\end{Verbatim} 
Hier wird ein Knoten vom Typ Person erzeugt. Es ist möglich, mehrere Knoten mit den dazugehörigen Relationen zu erzeugen. Indizes  für die Objekte oder Attribute von Objekten können ebenfalls über Create erzeugt werden.\newline
\textbf{Limit}: Beschränkt die Anzahl, welche durch das Return-Ausdruck zurückgegeben wird 
\begin{Verbatim}[frame=single]
MATCH (p:person) RETURN p.name LIMIT 10
\end{Verbatim}
Hier werden nur die Namen  von 10 Personen zurückgegeben\newline
\textbf{SUM/COUNT/AVG}: Wie in SQL wird die Summe, die Anzahl oder der Durchschnitt von einer gegebenen Menge gebildet. Solch eine Funktionen ist wird Aggregationsfunktion bezeichnet. \newline 
Durch die gegebenen Prädikate wird beeinflusst welche Daten mittels des Musters gesucht und zurückgegeben werden. Es besteht keine Möglichkeit, die Art der Berechnung zu beeinflussen. Cypher wird dadurch als nutzerfreundlichere aber auch weniger performante Alternative zu den APIs empfohlen \parencite{vukotic2015neo4j}. In dieser Evaluation wird sowohl Cypher als auch die APIs für einen direkten Vergleich verwendet. Cypher wird in Neo4j ausschließlich auf einem Prozessorkern ausgeführt. Durch das Einbinden von User Defined Functions (UDF) für Cypher ist das Ausführen auf mehreren Prozessorkernen in einigen Fällen möglich. Diese UDFs können von den Nutzer erstellt und zur Verfügung gestellt werden und  durch Bibliotheken wie die Awesome Procedures on Cypher (APOC) verbreitet werden \parencite{APOC}.

\subsubsection{Java Core API}
Als nahe Schnittstelle zu den Kernfunktionen von Neo4j bietet die Java Core API die meiste Kontrolle über das System und besitzt bei korrekter Verwendung eine bessere Performanz als Cypher \parencite{vukotic2015neo4j}. Zur Verwendung dieser imperativen API sind weitreichende Programmierkenntnisse und Wissen über die Neo4j-Bibliotheken, sowie ein genaues Verständnis  über die Daten in dem Graph erforderlich. Wenn diese Kenntnisse gegeben sind, ist die API flexible verwendbar  und der Nutzer hat einen hohen Einfluss darauf, wie die Anfragen bearbeitet werden sollen und kann eine optimale Berechnungsstrategie angeben \parencite{vukotic2015neo4j}. Der Nutzer gibt jede Transaktion explizit an, so ist es dem System möglich die Transaktionen zu parallelisieren und nicht ausgelastete Prozessorkerne zu nutzen. Gegeben sei folgende Transaktion: \newline
\begin{Verbatim}[frame=single,numbers=left,xleftmargin=5mm]
try ( Transaction tx = graphDb.beginTx() ){ 
	Node Peter = graphDb.getNodeById(Peter_ID);
	Set<Node> friends = new HashSet<Node>();
	for (Relationship R : Peter.getRelationships(FRIEND)) {  
	  Node friend = R.getOtherNode(userJohn);
	  friends.add(friend);
	}
	for (Node friend : friends) {
	  logger.info("Gefundener Freund: "
	   + friend.getProperty("name")); 
	}
}
\end{Verbatim}
\noindent Die gesamte Transaktion wird für eine Ausnahme sichere Programmierung mit einem try-Ausdruck umschlossen. In Zeile 2 wird die Variable $Peter$ vom Typ Node erstellt und wird mit dem Knoten mit der ID $Peter\_ID$ initialisiert. In Zeile 3 wird ein Set vom Typ Node erstellt, welches die Ergebnismenge darstellt. Durch die Zeilen 4-7 werden alle Knoten, die über die Relation FRIEND mit dem Knoten Peter verbunden sind, zur Ergebnismenge hinzugefügt. In den Zeilen 8-11 wird über die Ergebnismenge  iteriert und  alle gefundenen Nachbarn werden zurückgegeben. 

\subsubsection{Traversal API}
Die deklarative Travesal API dient zum spezifizieren von Traversierungen im Graph, sie ist näher an den Kernfunktionen von Neo4j als Cypher und weiter entfernt als die Core API. Die Traversal API erlaubt einen Zugriff auf Neo4j, welcher  abstrakter als die Core API ist. Der Nutzer muss keine genaues Verständnis von den Daten im Graphen haben. Es wird eine Beschreibung zur Traversierung  definiert und diese wird auf einen Graphen anwenden. Gegeben sei folgende Traversierung:
\begin{Verbatim}[numbers=left,xleftmargin=5mm,frame=single]
private Traverser getFriends(final Node person)
{
	TraversalDescription td = graphDb.traversalDescription()
		.depthFirst()
		.relationships( RelTypes.friend, OUTGOING )
		.evaluator( Evaluators.toDepth(2) );
	return td.traverse( person );
}
\end{Verbatim}
Diese Traversierung stellt in Java eine eigene Funktion dar, welche einen Ausgangsknoten als Eingabeparameter erwartet. In Zeile 3 wird eine neue Beschreibung zur Traversierung erstellt. Mit Zeile 4 gibt wird die Tiefensuche als Suchalgorithmus für die Traversierung spezifiziert. Durch Zeile 5 wird angegeben, dass nur über ausgehende, friend-Relationen traversiert wird. In Zeile 6  wird die maximale Traversierungstiefe auf zwei limitiert. Der Ausdruck in Zeile 7 wendet die erstellte Beschreibung zur Traversierung auf den übergebenen Ausgangsknoten an und gibt das Ergebnis zurück. \newline
Beim Traversieren kann der  Nutzer zwischen drei grundsätzlichen Vorgehen wählen: Breitensuche, Tiefensuche und bidirektionale Traversierung, diese haben abhängig von der Struktur des Graphen und der gestellten Anfrage eine unterschiedliche Laufzeit \parencite{vukotic2015neo4j}. In  Abbildung \ref{fig:Search} werden durch die Pfeile die Reihenfolgen des jeweiligen Suchalgorithmus dargestellt. \newline
\textbf {Breadth-First-Search (Breiten-Suche)} : Zuerst werden alle Knoten mit derselben Distanz betrachtet, danach werden alle Knoten mit der nächst höheren Distanz betrachtet, dies wird solange ausgeübt bis alle Knoten betrachtet wurden. \newline
\textbf {Depth-First-Search (Tiefen-Suche)}: Zunächst wird ein Knoten der Tiefe 1 gewählt, ausgehend von diesem Knoten wird ein nächsttieferer Knoten, der noch nicht betrachtet wurden, gewählt. Dies wird so lange wiederholt bis keine neuen nächsttieferen Knoten hinzugefügt werden können, danach so lange rückwärts gegangen bis ein Knoten zu erreichen ist, der noch nicht betrachtet wurde. \newline
\textbf {Bidirektionale Traversierung}: Es werden zwei Knoten gewählt und von jedem diesen Knoten wird eine Traversierung  mit Tiefen- oder Breitensuche gestartet, die beiden Traversierungen müssen nicht über den gleichen Relationstypen verlaufen oder im selben Schrittintervall vollzogen werden. Bei dem Treffen der beiden Traversierungs-Pfade muss ein Verhalten definiert werden .
\FloatBarrier
\begin{figure}[!htb]
	\centering
	\includegraphics [width=14cm, height=6cm]{Figures/New_Search.png}
	\caption[Breiten- und Tiefensuche]{ eigenen Abbilung.}
	\label{fig:Search}
\end{figure} 
\FloatBarrier
\noindent Wenn der Nutzer mit der Struktur der Daten in dem Graph vertraut ist, kann die ausgewählte Methode einen erheblichen Performanzunterschied hervorbringen \parencite{vukotic2015neo4j}. 

\subsection{Anfragebearbeitung und Planoptimierung}
Nach Angaben von Neo4j werden Anfragen in Cypher, nach folgendem Muster bearbeitet \parencite{Optimizer}:
\begin{enumerate}
	\item Umwandeln der Eingabe in einen abstrakten Syntax Baum (ASB)
	\item Optimieren des ASB
	\item Erstellen eines Anfragegraphen aus dem ASB
	\item Erstellen eines logischen Plans
	\item Optimieren des logischen Plan 
	\item Erstellen eines Ausführungsplan aus dem logischen Plan
	\item Ausführen der Anfrage mit Hilfe des Ausführungsplans  
\end{enumerate}
Die Schritte 2-5 werden vom Cypher Query Optimizer übernommen. \newline \newline
1. Die Eingabe wird auf semantische Fehler bezüglich der Datentypen und auf allgemeine syntaktische Fehler überprüfst. Wenn keine Fehler erkannt wurden, wird die Eingabe in einen ASB umgewandelt, welcher die Semantik der Eingabe darstellt. \newline
\newline
2. Die Optimierung des ASB beinhaltet folgende Schritte: 
\begin{enumerate}[label=(\roman*)]
	\item Alle Bedingungen, die sich in einem \textbf{Match}-Prädikat befinden, werden in das \textbf{Where}-Prädikat verschoben
	\item  Semantisch-äquivalente \textbf{Where}-Prädikate werden zusammengefasst
	\item Ersetze alle Synonyme wie: \textbf{RETURN * => RETURN x as x, y as y}
	\item Fasse Konstanten zusammen wie: \textbf{3 + 3 => 6}
	\item Setze bei anonymen Knoten einen Namen ein  wie: \textbf{ MATCH () => MATCH (n:Person)}
	\item Ersetze das Gleichheitszeichen durch ein 'IN' wie: \textbf{MATCH (n) WHERE id(n) = 12 => MATCH n WHERE id(n) IN [12]}
\end{enumerate}
3. Durch das Erstellen eines Anfragegraphen wird ein abstraktere Darstellung für die Anfrage erzeugt.\newline \newline
4. Aus dem Anfragegraphen wird ein logischer Plane für jede Anfrage erzeugt. Dieser Plan ist ein Baum mit maximal zwei Kindern, welche die verwendeten Operatoren darstellen. Dies gleicht einem logischen Plan für relationale Datenbanken. Aus dem logischen Plan wird der geschätzte Bearbeitungsaufwand für eine Anfrage gelesen. Der Aufwand wird aus den benötigten Eingabe-/Ausgabe-Operatoren auf den Speicher oder Indizies und den durchzuführenden Traversierungen ermittelt. Bei jedem Durchgang werden mehrere Pläne für eine Anfrage erzeugt, der Optimierer wählt aus diesen Plänen mit einem gierigen Suchalgorithmus einen Plan aus, welcher nahe am optimalste Plan ist, aber garantiert nicht der langsamste Plan ist. \newline \newline
5. Nachdem die Pläne erstellt wurden und einer dieser Pläne ausgewählt wurde, wird der ausgewählte Plan nochmals optimiert. Alle Komponenten werden so weit wie möglich vereinfacht und Redundanzen werden zusammengeführt. Jede Art von Verschachtelung, wie ein Match-Prädikat in einer Where Bedingung wird aufgelöst. Die Arbeit des Optimierers ist mit dem 5. Schritt abgeschlossen \newline \newline
6. Der logische Plan gibt vor, wie die Anfrage semantisch bearbeitet wird, es wird nicht spezifiziert mit welchen konkreten Operatoren die Anfrage ausgeführt werden soll. Der Ausführungsplan baut auf dem logischen Plan auf und gibt für die benötigten Operatoren eine hierarchische Anordnung von physischen  Implementierungen vor. Dieser Plan wird durch die Database Engine erstellt.  \newline \newline
7. Die Operatoren in dem Ausführungsplan werden während der Laufzeit in der vorgegebenen Reihenfolge ausgeführt. 


\subsection{Speicherverwaltung in Neo4j}
Im Gegensatz zu relationalen Datenbanken wird der Speicher in Neo4j in sogenannten Record Files unterteilt \parencite{angles2012comparison}. Jede einzelne Datei speichert jeweils einen Teil des Graphen wie Knoten, Kanten, Attribute ab. Die Objekte besitzen in den Dateien eine feste Größe, dies erlaubt einen Zugriff in konstanter Zeit \parencite{robinson2013graph}. Wenn ein Knoten mit der ID 100 gesucht wird und ein einzelner Knoten X Bytes groß ist, wird der gesuchte Knoten bei Byte 100*X beginnen. Die genaue Größe  variiert, je nach betrachteter Neo4j Version, seit Neo4j Version 3 besitzt ein Knoten die Größe 15 Byte und werden in der Datei node-store gespeichert \parencite{Storage}. Die Objekte werden nach \parencite{robinson2013graph} wie folgt verwaltet:
\subsubsection{Verwaltung der Knoten im Speicher}
 Die Knoten werden im Knotenspeicher verwaltet. Das erste Byte eines Eintrages kennzeichnet, ob der Knoten verwendet wird bzw. genutzt werden kann. Die nachfolgenden 4 Bytes kennzeichnen die ID für die erste Relation, die mit dem Knoten verbunden ist. Die darauffolgenden 4 Bytes beschreiben das erste Attribut des Knoten. Die nächsten 5 Bytes verweisen auf ein Label, welches gegebenenfalls verwendet wurde und sich im Label-Store befindet. Das letzte Byte ist für bestimmte Flags und für zukünftige Verwendungszwecke. 
\subsubsection{Verwaltung der Kanten im Speicher}
Die Kanten werden in dem Relationsspeicher gespeichert. Jeder Eintrag besitzt die IDs zu den zugehörigen Knoten, einen Zeiger zu dem Relationstypen, welcher in dem Relationstypspeicher gespeichert ist, einen Zeiger zu den vorherigen und nächsten Relationen der beiden zugehörigen Knoten und ein Flag, das angibt ob die betrachtete Relation die erste in einer Liste von zugehörigen Relationen im Graphen ist. 
\subsubsection{Verwaltung der Attribute im Speicher}
Neo4j wird als Attributs Graphdatenbank bezeichnet, dort kann jeder Knoten und jede Kante  ein oder mehrere Attribute besitzen. Diese Attribute befinden sich im Attributsspeicher. Jeder Eintrag zu einem Attribut ist in 1-4 Blöcke und einer ID zu einem nächstfolgenden Attribut unterteilt. Jeder der Blöcke beinhaltet Informationen über einen Typen, welcher ein String, ein Array oder ein Java-Standard Typ sein kann und einen Zeiger auf ein Index zu dem Namen des Attributs.  
\subsubsection{Indizies}
Ein Index ist eine redundante Kopie von Daten in einer Datenbank, dadurch kann auf ein Element direkt zugegriffen werden, ohne dass über die gesamte Datenstruktur iteriert werden muss. Dieses Vorgehen beschleunigt die Suche nach Daten in der Datenbank. Neo4j besitzt die Indexfreie Nachbarschafts Eigenschaft, dadurch sind die Knoten nicht über einen Index  sondern direkt über Relationen verbunden. Bei einer GDB ohne diese Eigenschaft werden die Knoten durch einen globalen Index verbunden und eine weitere Datenstruktur  wie ein Binärbaum oder eine Hash-Tabelle wird für das Traversieren verwendet \parencite{robinson2013graph}. \newline
Neo4j nutzt Indizes ausschließlich für Attribute und erlaubt das Erstellen eines Index zu einem oder mehreren Attributen. Sobald ein Index zu einem angegebenen Attribute erstellt wurde, wird dieser immer aktuell gehalten \parencite{Index}.  	

\subsubsection{Caching im Speicher}
Neoj4 nutzt den LRU-K page-affined cache, wodurch die Datenbank aufgeteilt wird und eine feste Anzahl von Teilen der Datenbank im Hauptspeicher gehalten wird. Die Auswahl der Teile, die im Hauptspeicher gehalten werden, geschieht nach dem least frequently used (LFU) Vorgehen. Bei dieser Methode wird ein selten genutzter Teil aus dem Speicher entfernt und einer häufig genutzter Teil wird im Speicher gehalten \parencite{robinson2013graph}.
\subsection{CAP und ACID unter Neo4j}
Das CAP-Theorem charakterisiert das Verhalten einer Datenbank anhand von folgenden drei Eigenschaften: Konsistenz (Consistency), Verfügbarkeit (Availability), Partitionstoleranz (Partition Tolerance)  \parencite{simon2000brewer}. Konsistenz beschreibt die Eigenschaft, dass die Daten bei einer Parallelisierung zur selben Zeit dieselben Werte besitzen und das gleiche Verhalten aufweisen. Verfügbarkeit beschreibt die Möglichkeit, zu jeder Zeit eine Anfrage an das System stellen zu können und auch zu jeder Zeit eine Antwort auf die gestellte Anfragen bekommen zu können. Partitionstoleranz gewährleistet, dass sich das Verhalten des Systems nicht verändert, wenn das System in mehrere kleinere Teilstücke, auch  Partitionen genannt, zerteilt wird. Alle Partitionen müssen das gleiche Verhalten wie das gesamte System aufweisen \parencite{simon2000brewer}. Neo4j erfüllt die Bedingung der Verfügbarkeit und der  Partitionstoleranz \parencite{vukotic2015neo4j} und wird als “AP-Database” bezeichnet. \newline
Die ACID Eigenschaft setzt sich aus vier Bedinungen zusammen, die das Verhalten der Transaktionen einer  Datenbank beschreiben \parencite{haerder1983principles}. Atomar (Atomicity) beschreibt, dass jede Transaktion einzeln betrachtet wird und entweder komplett fehlschlagen oder erfolgreich sein kann. Durch Konsistenz (Consistency) kann jede Transaktion nur valide Daten verwenden und den validen Zustand einer Datenbank nicht in einen nicht-validen Zustand überführen. Isolation (Isolation) erwartet, dass jede Transaktion unabhängig von einer parallellaufenden Transaktion abläuft und keine dieser Transaktionen beeinflusst. Haltbarkeit (Durability) ist gegeben, wenn sich der Effekt einer Transaktion auf den Speicher ausübt und auch bei einem Absturz des Systems bestehen bleibt \parencite{haerder1983principles}. \newline
  Neo4j erfüllt die ACID Eigenschaft, wenn es auf einem einzelnen System ausgeführt wird \parencite{holzschuher2013performance}. Das atomische und haltbare Verhalten wird durch den write-ahead log versichert. Bei diesem Mechanismus  werden alle Operationen einer Transaktionen nach dem Beenden der Transaktion in einer Log-Datei  festgehalten, bevor diese  den Speicher beeinflussen, so kann auch bei einem Absturz des Systems die Log-Datei genutzt werden um ein vorherige Transaktion zu wiederholen. Diese Log-Datei wird auch für die  hohe Verfügbarkeits Funktion genutzt, welche es erlaubt, die Datenbank in einem Netzwerk auf mehrere Systeme zu verwenden, dennoch ist es nicht mehr möglich, ein  ACID Verhalten bei Verwendung der hohen Verfügbarkeits Funktion zu gewährleisten, da es keine absolute Garantie für ein  atomisches und konsistente Verhalten gibt \parencite{vukotic2015neo4j}. Um die Isolation und Konsistenz zu gewährleisten, werden Verklemmungen, bei der sich Transaktionen gegenseitig blockieren, verhindert. Neo4j erstellt hierfür vor einer Transaktion Schreib- und Lesesperren und verhindern, dass eine weitere Transaktion auf Daten zugreift bzw. diese überschreibt, die zeitgleich von dieser Transaktion verwendet werden. \parencite{raj2015neo4j}.
\section {Modi zur Bedienung von Neo4j}
Neben der Nutzung von Neo4j mit den Anwendungen Neo4j  Browser und Neo4j Desktop, lässt sich Neo4j mit der Programmiersprache Java im eingebetteten Modus und Server-Modus nutzen. Diese Modi beeinflussen wie die Bibliotheken von Neo4j, welche für die Bearbeitung der Anfragen benötigt werden, aufgerufen und verwendet werden. Dafür wird das Verhalten der Java virtual machine (JVM), welche die Kompatibilität des geschriebenen Java-Codes gewährleistet, angepasst. Der verwendete Speicher ist für beide Modi derselbe.

\subsection{Der eingebettete Modus}
Der eingebettete Modus ermöglicht die direkte Nutzung der  Bibliotheken von Neo4j mit einer Programmiersprache, die von der JVM unterstützt wird und für die ein Treiber zur Verfügung steht. Alle Bibliotheken von Neo4j werden durch den GraphDataBaseService von Neo4j verwaltet. Der eingebettete Modus wird für Anwendungen, die auf einem einzelnen System laufen, empfohlen \parencite{raj2015neo4j}. Da sowohl alle Funktionen von Neo4j als auch die Anfragen in der selben JVM agieren. Das System kann in diesem Modus nur von einem Nutzer verwendet werden, welcher dadurch die volle und alleinige Kontrolle über jede Transaktion hat und  jede zur Verfügung stehenden Funktion der API nutzen kann. Daraus resultiert, dass dieser Nutzer für ein sicheres Starten und Beenden der Datenbank in seiner Sitzung verantwortlich ist \parencite{robinson2013graph}. Der Modus wird in Abbildung \ref{fig:Embedded} dargestellt.
\FloatBarrier
\begin{figure}[!htb]
	\centering	
	\includegraphics [width=10cm, height=5cm]{Figures/embedded}
	\caption[Eingebetteter Modus]{aus https://livebook.manning.com/book/neo4j-in-action/chapter-10/9 (16.07.19)}
	\label{fig:Embedded}
\end{figure}
\FloatBarrier
\subsection{Der Server-Modus} \label{Server}
Im Server-Modus werden alle Anfragen in einem eigenen Prozess verwaltet und mittels HTTP (Hypertext Transfer Protocol) als JSON-formatiertes Dokument an die REST API übermittelt \parencite{robinson2013graph}. Die REST API läuft in der JVM von Neo4j und verwaltet alle eintreffenden Anfragen der Nutzer und gibt diese an den GraphDatabaseService weiter, welcher die von Neo4j zur Verfügung gestellten Bibliotheken verwaltet. \newline 
Der Server-Modus ermöglicht die Verwendung der hohe Verfügbarkeits Funktion, welche das Nutzen der Datenbank auf mehren Systemen erlaubt \parencite{raj2015neo4j}. Da die Anfragen der Nutzer getrennt von der JVM von Neo4j verwaltet werden und so mehrere Nutzer die Möglichkeit besitzen auf die Neo4j Datenbank zuzugreifen. Durch die  Übertragungsverzögerung innerhalb der Netzwerks und dem Übertragen durch eine weitere API ist die Verzögerung höher als die des eingebetteten Modus. Dieser Modus wird in Abbildung \ref{fig:Server} dargestellt.
\begin{figure}[!htb]
	\centering
	\includegraphics [width=10cm, height=5cm]{Figures/server}
	\caption[Server Modus]{aus https://livebook.manning.com/book/neo4j-in-action/chapter-10/9 (16.07.19)}
	\label{fig:Server}
\end{figure}

