% Chapter Template

\chapter{Grundlagen} % Main chapter title

\label{Kaptiel2} % Change X to a consecutive number; for referencing this chapter elsewhere, use \ref{ChapterX}

%----------------------------------------------------------------------------------------
%	SECTION 1
%----------------------------------------------------------------------------------------

\section{Graph}

Ein Graph ist eine abstrakte Datenstruktur, welche für die GDBs verwendet werden \parencite{vicknair2010comparison}. Ein Graph besteht aus einer endlichen nicht leeren Summe von Knoten, auch Vertex oder Punkte genannt und Kanten. Die Kanten des Graphen bildet die Verbindung zwischen zwei Knoten \url{ http://wwwmayr.in.tum.de/lehre/2008WS/ea/EA-7.pdf}(7.06.19), wenn die Kanten aus einem geordneten Paar bestehen und somit eine Richtung besitzen wird der Graph als gerichtet bezeichnet, bei einem ungeordneten Paar wird von einem ungerichteten Graph gesprochen\parencite{bondy1976graph}. Wenn die Kanten Attribute, wie zum Beispiel Kosten besitzen wird der Graph als gewichtet bezeichnet und ohne Attribute als ungewichtet  \url{ http://wwwmayr.in.tum.de/lehre/2008WS/ea/EA-7.pdf}(7.06.19). 

%-----------------------------------
%	SUBSECTION 1
%-----------------------------------
\section{Neo4j}

Neo4j ist eine Graphdatenbank, welche in Java implementiert wurde \parencite{vukotic2015neo4j}. Als Grundlegende Datenstruktur wird ein gerichteter und gewichteter  Graph verwendet.Knoten stellen die Entitäten,beispielsweise eine Person oder ein Produkt,  da und  Kanten stellen die Relationen zwischen den Entitäten da, beispielsweise “isAngestellt” und können optional ein Gewicht und eine Richtung besitzen. Attribute können als zusätzliche Informationen in den Knoten gespeichert  werden wie zum Beispiel: Name bei einem Knoten “Person”. Knoten können mit Bezeichnern versehen werden, um so leichter in Anfragen  verwendet werden zu können. Die Operationen sind entweder durch durch die  Anfragesprache  Cypher,welche eine standardisierten Syntax mit  mehreren vordefinierten Funktionen besitzt, limitiert oder durch die jeweilige verwendete Programmiersprache.Es wird das Einbetten weiterer Bibliotheken, welche weitere Funktionen oder Algorithmen stellen, unterstützt. Durch weitere Funktionen wie zum Beispiel die Bibliothek “APOC” \footnote{https://neo4j-contrib.github.io/neo4j-apoc-procedures/ (17.05.19) } ist es möglich, Daten aus verschiedenen Formaten wie JSON,XSL oder XML in die Datenbank zu laden oder Daten aus anderen Web-APIs zu nutzen. Neo4j lässt sich in einem  eingebetteten Modus oder einem  Server Modus nutzen. Der eingebettete Modus dient der direkten  Nutzung durch die Java Core API(Application programming interface)  von Neo4j. Der Server Modus ermöglicht eine getrennte Ausführung von dem Code und dem bestehenden Neo4j System.

%----------------------------------------------------------------------------------------
%	SECTION 2
%----------------------------------------------------------------------------------------

\section{CAP und ACID unter Neo4j}

Das CAP-Theorem charakterisiert  das Verhalten einer Datenbank anhand von folgenden 3 Eigenschaften: Consistency(Konsistenz), Availability(Verfügbarkeit), Partitionstoleranz \parencite{simon2000brewer}. Konsistenz beschreibt die Eigenschaft, dass die Daten in allen Partitionen zur selben Zeit dieselben Werte besitzen und das gleiche Verhalten aufweisen. Verfügbarkeit beschreibt die Möglichkeit zu jeder Zeit eine Anfrage an das System stellen zu können und auch zu jeder Zeit eine Antwort auf die gestellte Anfragen bekommen zu können. Partitionstoleranz gewährleistet, dass sich das Verhalten des System nicht verändert, wenn mehrere Partitionen von diesem System erstellt werden und alle Partitionen müssen das gleiche Verhalten aufweisen  \parencite{simon2000brewer}. Neo4j erfüllt die Bedingung der Verfügbarkeit und der  Partitionstoleranz[16] und wird so als “AP-Database” bezeichnet. \newline
Die ACID Eigenschaft setzt sich aus 4 Eigenschaften zusammen, die das Verhalten der Transaktionen einer  Datenbank beschreiben \parencite{haerder1983principles}. Atomicy(Atomisch) beschreibt, dass jede Transaktion einzeln betrachtet wird und nur fehlschlagen oder erfolgreich sein kann. Durch Consistency(Konsistenz) kann jede Transaktion nur valide Daten verwenden und den validen Zustand einer Datenbank nicht in einen nicht-validen Zustand überführen. Isolation erwartet, dass jede Transaktion unabhängig von einer parallel-laufenden Transaktion abläuft und keine dieser Transaktionen beeinflusst. Durability(Haltbarkeit) ist gegeben, wenn der Effekt einer Transaktion auf den Speicher ausgeübt wurde und auch bei einem Absturz des Systems beibehalten wird \parencite{haerder1983principles}. Solange Neo4j auf einem einzelnen System läuft und nicht das High Aviability Feature der Enterprise Edition nutzt, ist es ACID konform \parencite{holzschuher2013performance}. Das atomische und haltbare Verhalten wird durch das sogenannte write-ahead log(wat) versichert. Bei diesem Mechanismus  werden alle Operationen einer Transaktionen nach dem Beenden der Transaktion in einem Log-File  festgehalten, bevor diese  den Speicher beeinflussen, so kann auch bei einem Absturz des System das Log-File genutzt werden um ein vorherige Transaktion zu wiederholen.  Dieses Log-File wird auch für die High-Availability  genutzt, welche es erlaubt die Datenbank in einem Netzwerk auf mehrere Systeme zu verwenden, dennoch ist es nicht mehr möglich ein  ACID Verhalten zu gewährleisten, da keine absolute Garantie für ein  atomisches und konsistente Verhalten gibt \parencite{vukotic2015neo4j}. Eine weitere versicherung für das atomische Verhalten bildet das Verhindern von Deadlocks innerhalb der Transaktion. Zur Verhinderung von Deadlocks wird “RWLock” verwendet, was eine Implementierung des Java “ReentrantReadWriteLock” für Neo4j ist. Dieser verwaltet alle Schreibsperren, die während einer Transaktion erstellt werden und versucht potentielle Deadlocks zu erkennen \parencite{raj2015neo4j}.

%-----------------------------------
%	SUBSECTION 3
%-----------------------------------



