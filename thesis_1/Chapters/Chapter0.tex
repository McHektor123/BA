% Chapter 0

\chapter{Einführungs} % Main chapter title

\label{Kaptiel1} % For referencing the chapter elsewhere, use \ref{Chapter1} 

%----------------------------------------------------------------------------------------

% Define some commands to keep the formatting separated from the content 
\newcommand{\keyword}[1]{\textit{#1}}
\newcommand{\tabhead}[1]{\textbf{#1}}
\newcommand{\code}[1]{\texttt{#1}}
\newcommand{\file}[1]{\texttt{\bfseries#1}}
\newcommand{\option}[1]{\texttt{\itshape#1}}

Um Daten in einer Datenbank abzulegen, ist ein Datenmodell nötig, welches  das allgemeine Verhalten der Datenbank definiert. Dieses ist durch folgende Eigenschaften definiert: ein Liste von Datenstrukturtypen, ein Liste von Operatoren, die auf die Daten angewandt werden können und Regeln zur Vollständigkeit der Datenbank \parencite{codd1981data}. Zwei dieser Datenmodelle sind der Relationale-Ansatz und die NoSQL-Bewegung. Die meisten Datenbanken werden heutzutage nach dem relationalen Datenmodell verwaltet und mittels Structured Query Language (SQL) bearbeitet \parencite{miller2013graph}. Dieser Ansatz wurde über viele Jahre lang optimiert und gilt für viele Daten als performanteste Implementierung \parencite{miller2013graph}. Die Datenstruktur wird eine Tabelle verwendet, eine Reihe bildet ein Objekt und die Spalten bilden die dazugehörigen Attribute \parencite{miller2013graph}. Die Möglichkeiten der Operationen, mit denen die Daten verändert oder angefragt werden, sind durch den SQL-Standard bzw. die Implementierung des  Standards limitiert. Die Regeln zur Vollständigkeit der Datenbank sind ebenfalls im Standard festgehalten. \newline 
Als eine Alternative zu diesem relationalen Datenmodell gibt es die NoSQL-Bewegung, welche erstmals 1998 erwähnt wurde \parencite{strauch2011nosql}. Diese Bewegung versuchte zunächst den Gebrauch von SQL als Anfragesprache zu vermeiden und brachte in den folgenden Jahren weitere Datenmodelle hervor. Eines dieser Modelle ist das Darstellen von Daten in einem Graphen als Datenstruktur \parencite{miller2013graph}. Die sogenannten Graphdatenbanken (GDB) werden besonders zum Darstellen von Netzwerken verwendet \parencite{han2011survey}. Da das relationalen Datenmodell bei großen Datenmengen und vielen komplex verbundene Informationen durch eine hohe Anzahl von notwendigen Joins nicht performant verwendet werden kann \parencite{miller2013graph}. Für GDBs gibt es keine standardisierte Anfragesprache und  die Menge an möglichen Operatoren ist verschieden.  \newline
Eine der populärsten Graphdatenbank ist das open-source Projekt Neo4j\parencite{francis2018cypher}. In dieser Arbeit werden die Eigenschaften von Neo4j als Datenbank managment System beschrieben und die verwendete Architektur wird dargestellt.Aufbauend auf diesen Angaben werden Annahmen über die Performanz des System getroffen und mittels selbst erstellter Anfragen werden diese überprüft. Aufbauend auf dem analysierten Verhalten wird eine abschließende Bewertung zur dem Databank managment System Neo4j gegeben. 

