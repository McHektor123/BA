% Chapter 0

\chapter{Einführung} % Main chapter title

\label{Kaptiel1} % For referencing the chapter elsewhere, use \ref{Chapter1} 

%----------------------------------------------------------------------------------------

% Define some commands to keep the formatting separated from the content 
\newcommand{\keyword}[1]{\textit{#1}}
\newcommand{\tabhead}[1]{\textbf{#1}}
\newcommand{\code}[1]{\texttt{#1}}
\newcommand{\file}[1]{\texttt{\bfseries#1}}
\newcommand{\option}[1]{\texttt{\itshape#1}}

Um Daten in einer Datenbank abzulegen, ist ein Datenmodell nötig, welches das allgemeine Verhalten der Datenbank definiert. Dieses Modell ist durch folgende Eigenschaften definiert: Ein Liste von Datenstrukturtypen, ein Liste von Operatoren, die auf die Daten angewendet werden können und Regeln zur Vollständigkeit der Datenbank \parencite{codd1981data}. Zwei dieser Datenmodelle sind der relationale Ansatz und die NoSQL-Bewegung. Der relationale Ansatz besitzt eine allgemeine hohe Effizienz und die NoSQL-Bewegung ist für spezifische Anwendungsfälle gut geeeignet, dadurch zählen diese beiden zu den verbreitetsten Datenmodellen \parencite{vicknair2010comparison}. \newline
Die meisten Datenbanken werden nach dem relationalen Ansatz verwaltet und mittels Structured Query Language (SQL) bearbeitet. Dieses Datenmodell wurde über viele Jahre optimiert und gilt für viele Daten als performanteste Implementierung \parencite{miller2013graph}. Als Datenstruktur wird eine Tabelle verwendet, wobei eine Reihe ein Objekt  und die Spalten die dazugehörigen Attribute bilden \parencite{tatarinov2002storing}. Die Menge der Operationen, mit denen die Daten verändert oder angefragt werden, ist an den verwendeten SQL-Standard bzw. eine Implementierung des Standards orientiert. Die Regeln zur Vollständigkeit der Datenbank hängen ebenfalls mit dem verwendeten SQL-Standard zusammen.  \newline 
Als eine Alternative zu dem relationalen Datenmodell gibt es die NoSQL-Bewegung, welche erstmals 1998 erwähnt wurde \parencite{NoSQL}. Diese Bewegung versuchte zunächst, den Gebrauch von SQL als Anfragesprache zu vermeiden, und brachte in den folgenden Jahren weitere Datenmodelle hervor. Eines dieser Modelle ist das Darstellen von Daten mit der Verwendung eines Graphen als Datenstruktur \parencite{miller2013graph}. Die sogenannten Graphdatenbanken (GDBs) werden unter anderem zum Darstellen von Beziehungen von Personen innerhalb eines sozialen Netzwerkes wie Facebook verwendet \parencite{han2011survey}. In der Modellierung eines solchen Netzwerkes bestehen viele Relationen zwischen den Personen und jede Person besitzt viele Eigenschaften wie Name oder Geburtsdatum. Das relationalen Datenmodell kann bei großen Netzwerken durch eine hohe Anzahl von notwendigen Berechnungen wie zum Beispiel Verbund nicht performant verwendet werden, GDBs besitzen für solche Netzwerke passendere Berechnungen, wie eine effiziente Traversierung über den Graphen und eigenen sich besser in solchen Modellierungen \parencite{miller2013graph}. Für GDBs gibt es keine standardisierte Anfragesprache und  die Menge an möglichen Operatoren, sowie die Regeln zur Vollständigkeit sind verschieden.
Eine der populärsten Graphdatenbanken ist die erwähnte Datenbank Neo4j\parencite{francis2018cypher}.  \newline 
In dem folgenden Kapitel dieser Arbeit wird eine Definition für die Datenstruktur Graph dargelegt und die Verwendung dieser Datenstruktur durch Neo4j wird beschrieben. Es wird ein Überblick von den zur Verfügung stehenden Werkzeugen und der grundlegenden Architektur geschaffen. Anschließend werden die einzelnen Bestandteile der Architektur, die zusammen das Datenbankmanagementsystem bilden, genauer beschrieben. Abschließend wird auf die Modi eingegangen, in denen Neo4j bedient werden
kann. \newline
Das Kapitel 3 beschreibt den verwendeten Datensatz und stellt eine Kategorisierung
für Anfragen vor. Aufbauend auf diesen Kategorien werden drei Testläufe vorgestellt, welche aus mehreren Anfragen bestehen. Es werden verschiedene Aspekte
bei diesen Anfragen betrachtet und es werden Hypothesen über das Verhalten der
Anfragen aufgestellt. \newline
Für Kapitel 4 werden alle Anfrage aus Kapitel 3 ausgeführt und die Ergebnisse werden tabellarisch aufgeführt. Mit Hilfe der Ergebnisse folgt eine Überprüfung der
aus Kapitel 3 aufgestellten Hypothesen. Anschließend wird ein praktische Anwendungsszenario für Neo4j beschrieben. Aufbauend auf den erfassten Erkenntnissen
wird unter Berücksichtigung der Limitierungen eine Bewertung über Neo4j getroffen.

