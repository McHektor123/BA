% Chapter 0

\chapter{Einführung} % Main chapter title

\label{Kaptiel1} % For referencing the chapter elsewhere, use \ref{Chapter1} 

%----------------------------------------------------------------------------------------

% Define some commands to keep the formatting separated from the content 
\newcommand{\keyword}[1]{\textit{#1}}
\newcommand{\tabhead}[1]{\textbf{#1}}
\newcommand{\code}[1]{\texttt{#1}}
\newcommand{\file}[1]{\texttt{\bfseries#1}}
\newcommand{\option}[1]{\texttt{\itshape#1}}

Um Daten in einer Datenbank abzulegen, ist ein Datenmodell nötig, welches  das allgemeine Verhalten der Datenbank definiert. Dieses Modell ist durch folgende Eigenschaften definiert: Ein Liste von Datenstrukturtypen, ein Liste von Operatoren, die auf die Daten angewendet werden können und Regeln zur Vollständigkeit der Datenbank \parencite{codd1981data}. Der Relationale-Ansatz und der NoSQL-Ansatz sind zwei verbreitetsten Datenmodelle. \newline
Die meisten Datenbanken werden heutzutage nach dem relationalen Datenmodell verwaltet und mittels Structured Query Language (SQL) bearbeitet \parencite{miller2013graph}. Dieser Ansatz wurde über viele Jahre optimiert und gilt für viele Daten als performanteste Implementierung \parencite{miller2013graph}. Als Datenstruktur wird eine Tabelle verwendet, eine Reihe bildet ein Objekt und die Spalten bilden die dazugehörigen Attribute \parencite{miller2013graph}. Die Möglichkeiten der Operationen, mit denen die Daten verändert oder angefragt werden, sind durch des SQL-Standard bzw. eine Implementierung des Standards limitiert. Die Regeln zur Vollständigkeit der Datenbank sind ebenfalls im Standard festgehalten. \newline 
Als eine Alternative zu diesem relationalen Datenmodell gibt es die NoSQL- Bewegung, welche erstmals 1998 erwähnt wurde \parencite{strauch2011nosql}. Diese Bewegung versuchte zunächst den Gebrauch von SQL als Anfragesprache zu vermeiden und brachte in den folgenden Jahren weitere Datenmodelle hervor. Eines dieser Modelle ist das Darstellen von Daten in einem Graphen als Datenstruktur \parencite{miller2013graph}. Die sogenannten Graphdatenbanken (GDBs) werden besonders zum Darstellen von Netzwerken verwendet \parencite{han2011survey}. Ein Netzwerk zeichnet sich durch eine hohe Anzahl an Verbindungen zwischen den einzelnen Entitäten aus, welche viele Eigenschaften besitzen können. Da das relationalen Datenmodell bei großen Datenmengen durch eine hohe Anzahl von notwendigen Berechnungen wie zum Beispiel Joins, nicht performant verwendet werden kann, eigenen sich GDBs für solche Szenario besser \parencite{miller2013graph}. Für GDBs gibt es keine standardisierte Anfragesprache und  die Menge an möglichen Operatoren ist verschieden.
Eine der populärsten Graphdatenbank ist das open-source Projekt Neo4j\parencite{francis2018cypher}.  \newline In dieser Arbeit wird die Performanz des Datenbank Managment System Neo4j untersucht. Für diese Evaluation wird ein selbsterstellter Datensatz generiert und es werden mehrere Anfragen wiederholt an die Systeme OrientDB und Neo4j gestellt. Für beide Systeme wird aus den benötigten Zeiten zur Bearbeitung der Anfragen das arithmetisches Mittel gebildet. Die ermittelten Werte für die beiden Systeme werden miteinander verglichen und es wird die Performanz von Neoj4 anhand dieses Vergleiches eingeschätzt. Zur Analyse des Verhaltens innerhalb von Neo4j werden alle Anfragen umformuliert, ohne ihre Semantik zu verändern. Die Unterschiede bei den benötigten Berechnungszeiten für semantische äquivalente Anfragen ermöglichen eine Beschreibung der Funktionsweise von Neo4j. 

