% Chapter 0

\chapter{Einführung} % Main chapter title

\label{Kaptiel1} % For referencing the chapter elsewhere, use \ref{Chapter1} 

%----------------------------------------------------------------------------------------

% Define some commands to keep the formatting separated from the content 
\newcommand{\keyword}[1]{\textit{#1}}
\newcommand{\tabhead}[1]{\textbf{#1}}
\newcommand{\code}[1]{\texttt{#1}}
\newcommand{\file}[1]{\texttt{\bfseries#1}}
\newcommand{\option}[1]{\texttt{\itshape#1}}

Um Daten in einer Datenbank abzulegen, ist ein Datenmodell nötig, welches das allgemeine Verhalten der Datenbank definiert. Dieses Modell ist durch folgende Eigenschaften definiert: Ein Liste von Datenstrukturtypen, ein Liste von Operatoren, die auf die Daten angewendet werden können und Regeln zur Vollständigkeit der Datenbank \parencite{codd1981data}. Der Relationale-tz und der NoSQL-Ansatz sind zwei der verbreitetsten Datenmodelle \parencite{vicknair2010comparison}. \newline
Die meisten Datenbanken werden heutzutage nach dem relationalen Datenmodell verwaltet und mittels Structured Query Language (SQL) bearbeitet. Dieser Ansatz wurde über viele Jahre optimiert und gilt für viele Daten als performanteste Implementierung \parencite{miller2013graph}. Als Datenstruktur wird eine Tabelle verwendet, wobei eine Reihe ein Objekt  und die Spalten dazugehörigen Attribute der Objekte bilden \parencite{tatarinov2002storing}. Die Menge der Operationen, mit denen die Daten verändert oder angefragt werden, ist an den verwendeten SQL-Standard bzw. eine Implementierung des Standards orientiert. Die Regeln zur Vollständigkeit der Datenbank hängen ebenfalls mit SQL-Standard zusammen.  \newline 
Als eine Alternative zu diesem relationalen Datenmodell gibt es den NoSQL-Ansatz, welcher erstmals 1998 erwähnt wurde \parencite{strauch2011nosql}. Dieser Ansatz versuchte zunächst, den Gebrauch von SQL als Anfragesprache zu vermeiden, und brachte in den folgenden Jahren weitere Datenmodelle hervor. Eines dieser Modelle ist das Darstellen von Daten in einem Graphen als Datenstruktur \parencite{miller2013graph}. Die sogenannten Graphdatenbanken (GDBs) werden besonders zum Darstellen von Netzwerken verwendet \parencite{han2011survey}. Ein Netzwerk zeichnet sich durch eine hohe Anzahl an Verbindungen zwischen den einzelnen Objekten aus, welche teilweise viele Eigenschaften besitzen. Da das relationalen Datenmodell bei großen Netzwerken durch eine hohe Anzahl von notwendigen Berechnungen wie zum Beispiel Verbund, nicht performant verwendet werden kann, eigenen sich GDBs für solche Netzwerke besser \parencite{miller2013graph}. Für GDBs gibt es keine standardisierte Anfragesprache und  die Menge an möglichen Operatoren ist verschieden.
Eine der populärsten Graphdatenbank ist die vorgestellte open-source Datenbank Neo4j\parencite{francis2018cypher}.  \newline 
Das folgende Kapitel dieser Arbeit, legt eine Definition für die Datenstruktur Graph dar und beschreibt die
Verwendung dieser Datenstruktur durch Neo4j genauer. Es wird
ein Überblick von den zur Verfügung stehenden Werkzeugen und der grundlegenden Architektur geschaffen. Anschließend werden die einzelnen Bestandteile der Architektur, die zusammen das Datenbankmanagementsystem bilden, genauer beschrieben. Abschließend wird auf die Modi eingegangen, in denen Neo4j bedient werden
kann. \newline
Das Kapitel 3 beschreibt den verwendeten Datensatz und stellt eine Kategorisierung
für Anfragen vor. Aufbauend auf diesen Kategorien werden 3 Testläufe vorgestellt, welche aus mehreren Anfragen bestehen. Es werden verschiedene Aspekte
bei diesen Anfragen betrachtet und es werden Hypothesen über das Verhalten der
Anfragen aufgestellt. \newline
Für Kapitel 4 werden alle Anfrage aus Kapitel 3 ausgeführt und die Ergebnisse werden tabellarisch aufgeführt. Mit Hilfe der Ergebnisse folgt eine Überprüfung der
aus Kapitel 3 aufgestellten Hypothesen. Anschließend wird ein praktische Anwendungsszenario für Neo4j beschrieben. Aufbauend auf den erfassten Erkenntnissen
wird unter Berücksichtigung der Limitierungen eine Bewertung über Neo4j getroffen.

